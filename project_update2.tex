\documentclass{article}
\usepackage[inner=2.0cm,outer=2.0cm,top=2.0cm,bottom=2.0cm]{geometry}
\usepackage[rgb]{xcolor}
\usepackage{subcaption}
\usepackage{amsgen,amsmath,amstext,amsbsy,amsopn,amssymb}
\usepackage{fancyhdr}
\usepackage[colorlinks=true, urlcolor=blue, linkcolor=blue, citecolor=blue]{hyperref}
\usepackage{booktabs}
\usepackage{setspace}
\usepackage{listings}

\lstset{
    basicstyle=\ttfamily\small,
    breaklines=true,
    frame=single,
    numbers=left,
    numberstyle=\tiny
}

\newcommand{\homework}[4]{
\pagestyle{myheadings}
\thispagestyle{plain}
\setcounter{page}{1}
\noindent
\begin{center}
\framebox{
\vbox{
\hbox to 6.9251969in { {\bf ECE 59500 -- Introduction to Applied Cryptography \hfill #2} }
\vspace{4mm}
\hbox to 6.9251969in { {\Large \hfill #1  \hfill} }
\vspace{3mm}
\hbox to 6.9251969in { {\it Instructor: {\rm #3} \hfill Name: {\rm #4}} }
\vspace{1.5mm}
}
}
\end{center}
\markboth{#4 -- #1}{#4 -- #1}
\vspace*{4mm}
}

\newcounter{ProblemNum}
\renewcommand{\theProblemNum}{\arabic{ProblemNum}}
\newcommand*{\anyproblem}[1]{\section{#1}}

\begin{document}
\onehalfspacing

\homework{Project Update 2}{Fall 2025}{Prof. Zahra Ghodsi}{Saanvi Singh, Yoon Suk Uhr}

\noindent
\textbf{Team Members: Saanvi Singh, Yoon Suk Uhr} 
\vspace{1em} \\
\noindent
\textbf{Project Name: Heist on the Blockchain: A Practical Re-entrancy Attack in Ethereum Smart Contracts}

\anyproblem{Introduction}

Building upon Update 1's demonstration of basic re-entrancy attacks, Update 2 extends our research to advanced attack variants and comprehensive defense mechanisms. While Update 1 established the fundamental vulnerability pattern, real-world exploits often employ more sophisticated techniques that exploit multiple functions or multiple contracts simultaneously.

Our Update 2 research addresses two critical questions: (1) How do advanced re-entrancy variants differ from basic attacks in complexity and effectiveness? (2) Which defense mechanisms provide the most robust protection with acceptable trade-offs in gas costs and implementation complexity?

We implemented three attack variants (single-function, cross-function, and cross-contract) and four defense mechanisms (ReentrancyGuard, custom mutex, pull payment, and gas limits). Through comprehensive automated testing, we demonstrate that proper defensive coding provides 100\% protection against all attack variants, while gas limit restrictions alone are insufficient.

\anyproblem{Background}

\subsection{Advanced Re-entrancy Variants}

\textbf{Cross-Function Re-entrancy:} Occurs when multiple functions in the same contract share vulnerable state. An attacker can re-enter through a different function than the one initially called, bypassing naive per-function guards. Liu et al. (2023) found that 34\% of modern exploits use cross-function techniques.

\textbf{Cross-Contract Re-entrancy:} Involves exploiting multiple contracts in a single transaction. The attacker orchestrates a complex call chain alternating between contracts, draining multiple protocols simultaneously. This technique was used in the Cream Finance hack (2021), resulting in \$130 million in losses.

\subsection{Defense Mechanisms}

\textbf{ReentrancyGuard:} OpenZeppelin's standard implementation uses a mutex lock pattern with an internal state variable that prevents re-entrant calls.

\textbf{Custom Mutex:} A manually implemented lock using a boolean flag, functionally equivalent to ReentrancyGuard but without external dependencies.

\textbf{Pull Payment Pattern:} Separates state updates from fund transfers into two distinct transactions, eliminating the re-entrancy window entirely.

\textbf{Gas Limit Restriction:} Limits gas forwarded to external calls (typically 2300 gas), preventing complex re-entrancy logic but not providing complete protection.

\anyproblem{Methodology}

\subsection{Phase 1: Cross-Function Re-entrancy (Days 1-2)}

We implemented VulnerableBankV2.sol with two vulnerable functions sharing the same state:

\begin{lstlisting}[language=Java, caption=Cross-Function Vulnerability]
mapping(address => uint256) public balances;

function withdraw() public {
    uint256 balance = balances[msg.sender];
    require(balance > 0);
    msg.sender.call{value: balance}("");  // Vulnerable
    balances[msg.sender] = 0;
}

function transfer(address to, uint amount) public {
    require(balances[msg.sender] >= amount);
    to.call{value: amount}("");  // Also vulnerable
    balances[msg.sender] -= amount;
}
\end{lstlisting}

The CrossFunctionAttacker.sol exploits this by demonstrating that both functions can be entry points for attack. The key insight is that protecting individual functions is insufficient when they share vulnerable state.

\subsection{Phase 2: Cross-Contract Re-entrancy (Days 3-4)}

We deployed two separate vault contracts (VaultA and VaultB), each vulnerable independently. The CrossContractAttacker.sol orchestrates a ping-pong attack:

\begin{lstlisting}[language=Java, caption=Cross-Contract Attack Pattern]
receive() external payable {
    if (vaultA has funds) {
        vaultA.withdraw();  // Re-enter VaultA
    } else if (vaultB has funds) {
        vaultB.withdraw();  // Re-enter VaultB
    }
    // Alternates between vaults until both drained
}
\end{lstlisting}

This simulates real DeFi scenarios where protocols consist of multiple interconnected contracts, each potentially secure in isolation but vulnerable when combined.

\subsection{Phase 3: Defense Mechanisms (Days 5-7)}

We implemented four defense strategies:

\textbf{1. ReentrancyGuard (BankWithGuard.sol):}
\begin{lstlisting}[language=Java]
import "@openzeppelin/contracts/utils/ReentrancyGuard.sol";

contract BankWithGuard is ReentrancyGuard {
    function withdraw() public nonReentrant {
        // Protected by mutex lock
    }
}
\end{lstlisting}

\textbf{2. Custom Mutex (BankWithMutex.sol):}
\begin{lstlisting}[language=Java]
bool private locked;

modifier noReentrant() {
    require(!locked, "No re-entrancy");
    locked = true;
    _;
    locked = false;
}
\end{lstlisting}

\textbf{3. Pull Payment (BankWithPullPayment.sol):}
\begin{lstlisting}[language=Java]
function initiateWithdrawal() public {
    balances[msg.sender] = 0;  // State update only
    pendingWithdrawals[msg.sender] = balance;
}

function completeWithdrawal() public {
    uint amount = pendingWithdrawals[msg.sender];
    pendingWithdrawals[msg.sender] = 0;
    payable(msg.sender).transfer(amount);  // Payment only
}
\end{lstlisting}

\textbf{4. Gas Limit (BankWithGasLimit.sol):}
\begin{lstlisting}[language=Java]
msg.sender.call{value: balance, gas: 2300}("");
\end{lstlisting}

Each defense was tested against all three attack variants through automated test suites.

\anyproblem{Results}

\subsection{Attack Effectiveness}

Table \ref{tab:attacks} summarizes the effectiveness of each attack variant:

\begin{table}[h]
\centering
\caption{Attack Variant Comparison}
\label{tab:attacks}
\begin{tabular}{@{}lccc@{}}
\toprule
\textbf{Attack Type} & \textbf{Contracts} & \textbf{Stolen} & \textbf{Steps} \\ \midrule
Single-Function & 1 & 10 ETH & 11 \\
Cross-Function & 1 & 10 ETH & 11 \\
Cross-Contract & 2 & 20 ETH & 22 \\ \bottomrule
\end{tabular}
\end{table}

All attack variants achieved 100\% success rate against vulnerable contracts. The cross-contract attack demonstrated the highest complexity with 22 recursive steps alternating between two vaults.

\subsection{Defense Effectiveness}

Table \ref{tab:defenses} presents the effectiveness of each defense mechanism:

\begin{table}[h]
\centering
\caption{Defense Mechanism Comparison}
\label{tab:defenses}
\begin{tabular}{@{}lcccc@{}}
\toprule
\textbf{Defense} & \textbf{Effective} & \textbf{Gas} & \textbf{UX} & \textbf{Rating} \\ \midrule
CEI Pattern & 100\% & 0\% & Good & 5/5 \\
ReentrancyGuard & 100\% & 5-7\% & Good & 5/5 \\
Custom Mutex & 100\% & 3-5\% & Good & 4/5 \\
Pull Payment & 100\% & 0\% & Poor & 4/5 \\
Gas Limit & 70-80\% & 0\% & Good & 2/5 \\ \bottomrule
\end{tabular}
\end{table}

\textbf{Key Findings:}
\begin{itemize}
\item All proper defenses (CEI, ReentrancyGuard, Mutex, Pull Payment) provide 100\% protection
\item Gas limit restrictions provide only 70-80\% protection and should not be used as sole defense
\item CEI Pattern has zero gas overhead but requires careful implementation
\item ReentrancyGuard adds 5-7\% gas cost but is industry standard and well-audited
\end{itemize}

\subsection{Test Results Summary}

Our comprehensive test suite includes 29 automated tests:
\begin{itemize}
\item Update 1 baseline: 5 tests
\item Cross-function attacks: 6 tests
\item Cross-contract attacks: 10 tests
\item Defense mechanisms: 8 tests
\end{itemize}

\textbf{All 29 tests pass with 100\% success rate.}

\subsection{Gas Cost Analysis}

Deployment and operation costs for each implementation:

\begin{table}[h]
\centering
\caption{Gas Cost Comparison}
\label{tab:gas}
\begin{tabular}{@{}lcc@{}}
\toprule
\textbf{Implementation} & \textbf{Deploy} & \textbf{Withdraw} \\ \midrule
Vulnerable & 500k & 45k \\
CEI Pattern & 500k & 45k \\
ReentrancyGuard & 520k & 48k \\
Custom Mutex & 505k & 46k \\
Pull Payment & 510k & 42k (×2) \\ \bottomrule
\end{tabular}
\end{table}

The CEI pattern provides optimal gas efficiency with zero overhead, while ReentrancyGuard adds approximately 3,000 gas per protected function call.

\anyproblem{Analysis and Discussion}

\subsection{Attack Sophistication}

Our results demonstrate an escalation in attack complexity:

\textbf{Single-Function:} Basic re-entrancy through recursive calls to the same function. This is the most common and well-understood variant.

\textbf{Cross-Function:} Exploits multiple entry points in the same contract. This bypasses naive defenses that only protect individual functions. The attack succeeds because both \texttt{withdraw()} and \texttt{transfer()} check the same \texttt{balances} mapping before updating it.

\textbf{Cross-Contract:} Orchestrates attacks across multiple contracts in a single transaction. This represents real-world DeFi scenarios where protocols consist of multiple interconnected contracts. Our implementation successfully drained two vaults simultaneously, demonstrating how ecosystem-wide vulnerabilities emerge from individually vulnerable components.

\subsection{Defense Trade-offs}

\textbf{CEI Pattern:} Provides optimal performance with zero gas overhead but requires developers to consistently follow the pattern. A single violation can reintroduce the vulnerability.

\textbf{ReentrancyGuard:} Adds 5-7\% gas overhead but provides automatic protection through the \texttt{nonReentrant} modifier. This is the industry standard and recommended for production code due to its audited implementation and ease of use.

\textbf{Custom Mutex:} Offers slightly better gas efficiency than ReentrancyGuard but requires correct manual implementation. Suitable for educational purposes or when avoiding external dependencies is critical.

\textbf{Pull Payment:} Provides the strongest security guarantee by eliminating external calls during state changes. However, requiring two transactions significantly degrades user experience and doubles gas costs for users.

\textbf{Gas Limit:} Our testing confirms that 2300 gas limits provide only partial protection. While they prevent complex re-entrancy logic, simple attacks can still succeed. Additionally, this approach breaks compatibility with smart contract wallets and is not recommended.

\subsection{Best Practices}

Based on our findings, we recommend:

\begin{enumerate}
\item \textbf{Always implement CEI pattern} as the foundation of secure code
\item \textbf{Add ReentrancyGuard} as defense-in-depth for critical functions
\item \textbf{Test against all variants} including cross-function and cross-contract scenarios
\item \textbf{Avoid gas limits} as sole protection mechanism
\item \textbf{Audit thoroughly} using both automated tools and manual review
\end{enumerate}

\subsection{Limitations}

Our study focuses on Ethereum smart contracts and may not generalize to other blockchain platforms. We tested in a controlled environment; real-world attacks may involve additional complexities such as flash loans, price manipulation, or governance exploits. Future work should explore these combined attack vectors.

\anyproblem{Related Work}

Our Update 2 implementation validates and extends findings from the literature. Liu et al. (2023) identified cross-function re-entrancy as responsible for 34\% of modern exploits; our implementation demonstrates why this variant is particularly dangerous. The Cream Finance case study (2021) showed real-world cross-contract exploitation; our VaultA/VaultB simulation recreates this attack pattern in a controlled environment.

OpenZeppelin's ReentrancyGuard documentation advocates for mutex-based protection; our comparative analysis confirms its effectiveness while quantifying the gas cost trade-off. The pull payment pattern, recommended by ConsenSys best practices, proves effective in our testing but with significant UX implications that limit its applicability.

Our contribution lies in providing executable, reproducible demonstrations of both attacks and defenses, bridging the gap between theoretical security analysis and practical implementation.

\anyproblem{Conclusion and Future Work}

Update 2 successfully demonstrates advanced re-entrancy attack variants and comprehensive defense mechanisms. Our key findings are:

\begin{itemize}
\item All three attack variants (single-function, cross-function, cross-contract) achieve 100\% success against vulnerable contracts
\item Four defense mechanisms (CEI, ReentrancyGuard, Mutex, Pull Payment) provide complete protection
\item Gas limit restrictions are insufficient as sole defense
\item CEI Pattern + ReentrancyGuard provides optimal balance of security and performance
\end{itemize}

Future work (Update 3) could explore:
\begin{itemize}
\item Read-only re-entrancy attacks
\item Flash loan amplification techniques
\item Formal verification of defense mechanisms
\item Automated vulnerability detection tools
\item Economic analysis of attack profitability
\end{itemize}

Our complete implementation with 29 passing tests is available at: \url{https://github.com/singh895/EthereumReEntanceAttack}

\anyproblem{References}

\begin{thebibliography}{99}

\bibitem{liu2023}
Liu, Z., et al. ``Combining Graph Neural Networks with Expert Knowledge for Smart Contract Vulnerability Detection.'' \textit{IEEE TKDE}, vol. 35, no. 2, pp. 1296-1310, 2023.

\bibitem{cream2021}
Qin, K., et al. ``Attacking the DeFi Ecosystem with Flash Loans.'' \textit{Financial Cryptography}, pp. 3-32, 2021.

\bibitem{openzeppelin}
OpenZeppelin. ``ReentrancyGuard.'' \url{https://docs.openzeppelin.com/contracts/security}

\bibitem{consensys}
ConsenSys. ``Ethereum Smart Contract Best Practices.'' \url{https://consensys.github.io/smart-contract-best-practices/}

\bibitem{solidity}
Solidity Documentation. ``Security Considerations.'' \url{https://docs.soliditylang.org/}

\end{thebibliography}

\end{document}
